\documentclass[12pt]{article}

\usepackage{fullpage}
\usepackage{fancybox} 
\usepackage{amssymb}
\usepackage{charter}
\usepackage{verbatim}
\usepackage{graphicx}
\setlength{\textwidth}{7in}
\setlength{\evensidemargin}{-0.24in}
\setlength{\oddsidemargin}{-0.24in}
\setlength{\textheight}{9.45in}
\setlength{\topmargin}{-0.45in}
\setlength{\parindent}{0.3in}
\headheight12pt
\headsep16pt
\pagestyle{myheadings}
\newcounter{ques}
\newenvironment{question}{\stepcounter{ques}{\noindent\bf Question \arabic{ques}:}}{\vspace{5mm}}

\begin{document} 

\begin{center} \large\bf
COMP 3804/MATH 3804\\
Design and Analysis of Algorithms  \\
Assignment 3
\end{center} 
\noindent \hrulefill

\vspace{5pt}
\noindent
{\bf Due Date: November  19th at 11:59PM}\\
\noindent
Your assignment should be submitted online on Brightspace as a single .pdf file.  The filename should contain your name and student number. No late assignments will be accepted.You can type your assignment or you can upload a scanned copy of it.  Please, use a good image capturing device. Make sure that your upload is clearly readable. If it is difficult to read, it will not be graded.\\

\vspace{1em} 


\begin{question}[20 points]\\ 
 We are given a directed graph $G =(V,E)$ with $|V|$  = $n$ vertices.  Let $goal$ be a vertex  of $G$.  We want to compute a shortest path   from each of  $k$ vertices of $G$ to $goal$, where $k<n$.
 
 \begin{itemize}
 	\item We could solve the problem by applying Dijkstra's algorithm $k$ times, ones for each of the $k$ starting vertices. What is the time  complexity (stated in terms of $n$ and $k$)? 
 	\item Alternately, we could  start at  the vertex $goal$ and  somehow go backwards to all $k$ vertices. Describe how this would work, i.e., how would we modify Dijkstra's algorithm and/or its input to achieve this?  Then, state the time complexity of this solution to our original problem. (Do not forget to argue why the algorithm, as modified, is correct!)
 \end{itemize}
\end{question} 

\begin{question}[15 points]\\  
	
	Let $G =(V,E)$  be a graph with vertex set,  $V$, and edge set $E$.  We would like to apply Topological Sort on $G$. One problem is that we do not know if $G$ is a DAG or not. What will happen if we apply the algorithm for  Topological Sorting on $G$ if $G$ is not a DAG?
	
	\end{question} 

\begin{question}[15 points]\\  
	
	Suppose we consider lattice paths from $(0,0)$ to $(n,n)$ on an  $n$  by  $n$ grid. The paths must, at every step,  either go up or  right. We call lattice path, $k$-Lpaths, if  they have precisely $2k$ path segments on one side of the diagonal  and the remaining $2(n-k)$ segments on the other.  Argue precisely why the number of 
	$k$-Lpaths is equal to the number of $(n-k)$-Lpaths. 
\end{question} 

\newpage


\begin{question}[15 points]\\  
Consider the graph given in Figure 1 above. 
\begin{itemize}
	\item Run DFS, from A, on the graph and classify each edge as being either: Tree edge, Forward edge, Back edge, or Cross edge. Show and argue: the algorithm execution,   pre(v) and post(v) time intervals and the edge-classification. (An edge type may or may not appear in a particular graph.)
	\item Find a topological order of the nodes or argue that no such order can exist. How does the DFS help detect that?
	\item Consider two intervals $[pre(u), post(u)]$ and $[pre(v), post(v)]$ for vertices $u$ and $v$, respectively. Argue precisely in your own words, why the intervals cannot overlap (other than if one is contained in the other).
\end{itemize}

.

\begin{figure}
	\centerline{\resizebox{!}{0.7\textwidth}{\includegraphics{DFSinput1.pdf}}}
	\caption{Input for DFS algorithm}
	\label{fig:DFSinput}
\end{figure}

\end{question}
\end{document} 
